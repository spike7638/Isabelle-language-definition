\documentclass[11pt,notitlepage,openany,oneside]{book}
\usepackage{fontspec}
\newcommand{\ignore}[1]{}
\setmonofont{JuliaMono}[
%  extension=.ttf,
  UprightFont=*-Regular,
  BoldFont=*-Bold,
  ItalicFont=*-RegularItalic,
  BoldItalicFont=*-BoldItalic,
]

\RequirePackage[svgnames]{xcolor}
\usepackage{listings}
\usepackage{isabelle-listings}
\makeatletter\lst@InputCatcodes
\def\lst@DefEC{%
 \lst@CCECUse \lst@ProcessLetter
^^80^^81^^82^^83^^84^^85^^86^^87^^88^^89^^8a^^8b^^8c^^8d^^8e^^8f%  ^^90^^91^^92^^93^^94^^95^^96^^97^^98^^99^^9a^^9b^^9c^^9d^^9e^^9f%
^^a0^^a1^^a2^^a3^^a4^^a5^^a6^^a7^^a8^^a9^^aa^^ab^^ac^^ad^^ae^^af%  ^^b0^^b1^^b2^^b3^^b4^^b5^^b6^^b7^^b8^^b9^^ba^^bb^^bc^^bd^^be^^bf%
^^c0^^c1^^c2^^c3^^c4^^c5^^c6^^c7^^c8^^c9^^ca^^cb^^cc^^cd^^ce^^cf%  ^^d0^^d1^^d2^^d3^^d4^^d5^^d6^^d7^^d8^^d9^^da^^db^^dc^^dd^^de^^df%
^^e0^^e1^^e2^^e3^^e4^^e5^^e6^^e7^^e8^^e9^^ea^^eb^^ec^^ed^^ee^^ef%  ^^f0^^f1^^f2^^f3^^f4^^f5^^f6^^f7^^f8^^f9^^fa^^fb^^fc^^fd^^fe^^ff%
  %^^^^20ac% Euro, deleted because it was causing problems
  ^^^^0153^^^^0152%  OE, and oe
  ^^^^2013^^^^2014% typo-dashes
  ^^^^2018^^^^2019^^^^201c^^^^201d% typo quotes
% Abbrevs (deleted because they all appear in punctuation)
% ZNotation (blackboard symbols removed, also punctuation, arrow, relation, and operator duplicates)
^^^^266f^^^^2a1f%
% Arrows
^^^^2190^^^^27f5^^^^290e^^^^21e0^^^^2192^^^^27f6^^^^290f^^^^21e2^^^^21d0^^^^27f8^^^^21da%
^^^^21d2^^^^27f9^^^^21db^^^^2194^^^^27f7^^^^21d4^^^^27fa^^^^21a6^^^^27fc^^^^2500^^^^2550%
^^^^21a9^^^^21aa^^^^21bd^^^^21c1^^^^21bc^^^^21c0^^^^21cc^^^^219d^^^^21c3^^^^21c2^^^^21bf%
^^^^2191^^^^21d1^^^^2193^^^^21d3^^^^2195^^^^21d5^^^^21a3^^^^2914^^^^2915^^^^21a0^^^^2900%
^^^^2916^^^^21f8^^^^21fb%
% Control
^^^^2759^^^^21e9^^^^21e7%
% Control Block
^^^^21d8^^^^21d9^^^^21d7^^^^21d6%
% Digit
^^^^^^01d7ec^^^^^^01d7ed^^^^^^01d7ee^^^^^^01d7ef^^^^^^01d7f0^^^^^^01d7f1^^^^^^01d7f2%
^^^^^^01d7f3^^^^^^01d7f4^^^^^^01d7f5^^^^00bc^^^^00bd^^^^00be%
% Document (with previously defined control removed)
^^^^2015^^^^2326^^^^2710^^^^21e4^^^^2508^^^^2509^^^^2501^^^^25aa^^^^25b8^^^^27a7%
^^^^204b^^^^25a9^^^^2b1a^^^^2217%
%Greek
^^^^03b1^^^^03b2^^^^03b3^^^^03b4^^^^03b5^^^^03b6^^^^03b7^^^^03b8^^^^03b9^^^^03ba%
^^^^03bb^^^^03bc^^^^03bd^^^^03be^^^^03c0^^^^03c1^^^^03c3^^^^03c4^^^^03c5^^^^03c6%
^^^^03c7^^^^03c8^^^^03c9^^^^0393^^^^0394^^^^0398^^^^039b^^^^039e^^^^03a0^^^^03a3%
^^^^03a5^^^^03a6^^^^03a8^^^^03a9%
%Icon
^^^^^^01f5cf^^^^^^01f5c0^^^^^^01f310^^^^^^01f4d3^^^^261b%
% Letters:
% Calligraphic
^^^^^^01d49c^^^^212c^^^^^^01d49e^^^^^^01d49f^^^^2130^^^^2131^^^^^^01d4a2^^^^210b%
^^^^2110^^^^^^01d4a5^^^^^^01d4a6^^^^2112^^^^2133^^^^^^01d4a9^^^^^^01d4aa^^^^^^01d4ab%
^^^^^^01d4ac^^^^211b^^^^^^01d4ae^^^^^^01d4af^^^^^^01d4b0^^^^^^01d4b1^^^^^^01d4b2%
^^^^^^01d4b3^^^^^^01d4b4^^^^^^01d4b5%
%Roman
^^^^^^01d5ba^^^^^^01d5bb^^^^^^01d5bc^^^^^^01d5bd^^^^^^01d5be^^^^^^01d5bf^^^^^^01d5c0%
^^^^^^01d5c1^^^^^^01d5c2^^^^^^01d5c3^^^^^^01d5c4^^^^^^01d5c5^^^^^^01d5c6^^^^^^01d5c7%
^^^^^^01d5c8^^^^^^01d5c9^^^^^^01d5ca^^^^^^01d5cb^^^^^^01d5cc^^^^^^01d5cd^^^^^^01d5ce%
^^^^^^01d5cf^^^^^^01d5d0^^^^^^01d5d1^^^^^^01d5d2^^^^^^01d5d3%
%Fraktur UC
^^^^^^01d504^^^^^^01d505^^^^212d^^^^^^01d507^^^^^^01d508^^^^^^01d509^^^^^^01d50a%
^^^^210c^^^^2111^^^^^^01d50d^^^^^^01d50e^^^^^^01d50f^^^^^^01d510^^^^^^01d511%
^^^^^^01d512^^^^^^01d513^^^^^^01d514^^^^211c^^^^^^01d516^^^^^^01d517^^^^^^01d518%
^^^^^^01d519^^^^^^01d51a^^^^^^01d51b^^^^^^01d51c^^^^2128%
%Frakture LC
^^^^^^01d51e^^^^^^01d51f^^^^^^01d520^^^^^^01d521^^^^^^01d522^^^^^^01d523%
^^^^^^01d524^^^^^^01d525^^^^^^01d526^^^^^^01d527^^^^^^01d528^^^^^^01d529%
^^^^^^01d52a^^^^^^01d52b^^^^^^01d52c^^^^^^01d52d^^^^^^01d52e^^^^^^01d52f%
^^^^^^01d530^^^^^^01d531^^^^^^01d532^^^^^^01d533^^^^^^01d534^^^^^^01d535%
^^^^^^01d536^^^^^^01d537%
%Blackboard
^^^^^^01d538^^^^^^01d539^^^^2102^^^^^^01d53b^^^^^^01d53c^^^^^^01d53d^^^^^^01d53e%
^^^^210d^^^^^^01d540^^^^^^01d541^^^^^^01d542^^^^^^01d543^^^^^^01d544^^^^2115%
^^^^^^01d546^^^^2119^^^^211a^^^^211d^^^^^^01d54a^^^^^^01d54b^^^^^^01d54c%
^^^^^^01d54d^^^^^^01d54e^^^^^^01d54f^^^^^^01d550^^^^2124%
% Logic
^^^^22a5^^^^22a4^^^^2227^^^^22c0^^^^2228^^^^22c1^^^^2200^^^^2203^^^^2204^^^^00ac%
^^^^25cb^^^^25a1^^^^25c7%
% Operator
^^^^21be^^^^22c4^^^^2229^^^^22c2^^^^222a^^^^22c3^^^^2294^^^^2a06^^^^2293^^^^2a05%
^^^^2216^^^^221d^^^^228e^^^^2a04^^^^00b1^^^^2213^^^^00d7^^^^00f7^^^^22c5^^^^22c6%
^^^^2219^^^^2218^^^^2295^^^^2a01^^^^2297^^^^2a02^^^^2299^^^^2a00^^^^2296^^^^2298%
^^^^2211^^^^220f^^^^2210^^^^222b^^^^222e^^^^00af^^^^2a3f^^^^2127^^^^291c^^^^2aa2%
^^^^2a21^^^^29f9^^^^2040%
% Punctuation
^^^^2237^^^^27e8^^^^27e9^^^^27ea^^^^27eb^^^^2308^^^^2309^^^^230a^^^^230b^^^^2987%
^^^^2988^^^^27e6^^^^27e7^^^^2983^^^^2984^^^^2989^^^^298a^^^^00ab^^^^00bb^^^^00a6%
^^^^2aff^^^^2026^^^^22ef^^^^2010^^^^00b7^^^^2a3e^^^^2981^^^^2032^^^^2039^^^^203a%
% Relation
^^^^22a2^^^^22a8^^^^22a9^^^^22ab^^^^22a3^^^^221a^^^^2264^^^^2265^^^^226a^^^^226b%
^^^^2272^^^^2273^^^^2a85^^^^2a86^^^^2208^^^^2209^^^^2282^^^^2283^^^^2286^^^^2287%
^^^^228f^^^^2290^^^^2291^^^^2292^^^^2260^^^^223c^^^^2250^^^^2243^^^^2248^^^^224d%
^^^^2245^^^^2323^^^^2261^^^^2322^^^^227a^^^^227b^^^^227c^^^^227d^^^^2225^^^^2016%
^^^^2af4^^^^2afd^^^^22b2^^^^22b3^^^^22b4^^^^22b5^^^^25c3^^^^25b9^^^^25b3^^^^225c%
^^^^2240^^^^25c1^^^^2a64^^^^25b7^^^^2a65^^^^2982^^^^22ff%
% Unsorted (Euro deleted because previously defined)
^^^^22c8^^^^2a1d^^^^2020^^^^2021^^^^221e^^^^2663^^^^2662^^^^2661^^^^2660^^^^2135%
^^^^2205^^^^2207^^^^2202^^^^266d^^^^266e^^^^2220^^^^00a9^^^^00ae^^^^00aa^^^^00ba%
^^^^00a7^^^^00b6^^^^00a1^^^^00bf^^^^00a3^^^^00a5^^^^00a2^^^^00a4^^^^00b0%
^^^^25ca^^^^2118^^^^00b4^^^^0131^^^^00a8^^^^00b8^^^^02dd^^^^03f5^^^^2311^^^^23ce%
^^^^2713^^^^2717^^^^2302^^^^2756%
% Search
^^00}
\lst@RestoreCatcodes\makeatother

\begin{document}

\chapter{What you get}
This project defines a language for the LaTeX `listings' package that allows 
you to include Isabelle source code in your document, with all the funny symbols
that Isabelle uses, and have it look much like it does in Isabelle's jedit window.
In particular, you should be able to cut-and-paste from Isabelle to your LaTeX
document and have things almost perfectly show up. Here's an example input:

\begin{verbatim}
\begin{lstlisting}
lemma times_assoc:
fixes a::nat and b and c
shows "P ≠ Q"
and "P ≠ l"
and "l∩ m = m"
and "∀P. P≤ P"
and "S ⟶ T"
and "⟦U⟧ ⟹ V" and "ℝ ≠ ℕ"
sorry
end
\end{lstlisting}
\end{verbatim}
\noindent
and the corresponding result:

\lstset{language=Isabelle,basicstyle=\ttfamily}
\begin{lstlisting}
lemma times_assoc:
fixes a::nat and b and c
shows "P ≠ Q"
and "P ≠ l"
and "l∩ m = m"
and "∀P. P≤ P"
and "S ⟶ T"
and "⟦U⟧ ⟹ V" and "ℝ ≠ ℕ"
sorry
end
\end{lstlisting}

And here's a list of all the characters that this language definition knows about, 
roughly corresponding to the various sections presented in the `Symbols' tab of
the jEdit interface, although if a symbol appears in two tabs, it only appears
once here, so that (for instance) the `Z notation' section has only a few symbols. 

\begin{lstlisting}

Digits

𝟬𝟭𝟮𝟯𝟰𝟱𝟲𝟳𝟴𝟵¼½¾𝟬𝟭𝟮𝟯

Abbrevs

›
‹

ZNotation

♯⨟

Arrows

←⟵⤎⇠→⟶

⤏⇢⇐⟸

⇚⇒⟹⇛↔⟷⇔⟺↦⟼

─═↩↪↽⇁↼⇀⇌↝

⇃⇂↿↑⇑↓⇓↕⇕

↣⤔⤕↠⤀⤖⇸⇻

Control

❙⇩⇧

Control Block

⇘⇙⇗⇖


Document

―⌦✐⇤┈┉━▪▸➧⁋▩⬚∗❙

Greek

αβγδεζηθικλμνξπρστυφχψωΓΔΘΛΞΠΣΥΦΨΩ

Icon

🗏🗀🌐📓☛

Letter

𝒜ℬ𝒞𝒟ℰℱ𝒢ℋℐ𝒥𝒦ℒℳ
𝒩𝒪𝒫𝒬ℛ𝒮𝒯𝒰𝒱𝒲𝒳𝒴𝒵

𝖺𝖻𝖼𝖽𝖾𝖿𝗀𝗁𝗂𝗃𝗄𝗅𝗆
𝗇𝗈𝗉𝗊𝗋𝗌𝗍𝗎𝗏𝗐𝗑𝗒𝗓

𝔄𝔅ℭ𝔇𝔈𝔉𝔊ℌℑ𝔍𝔎𝔏𝔐
𝔑𝔒𝔓𝔔ℜ𝔖𝔗𝔘𝔙𝔚𝔛𝔜ℨ

𝔞𝔟𝔠𝔡𝔢𝔣𝔤𝔥𝔦𝔧𝔨𝔩𝔪
𝔫𝔬𝔭𝔮𝔯𝔰𝔱𝔲𝔳𝔴𝔵𝔶𝔷

𝔸𝔹ℂ𝔻𝔼𝔽𝔾ℍ𝕀𝕁𝕂𝕃𝕄
ℕ𝕆ℙℚℝ𝕊𝕋𝕌𝕍𝕎𝕏𝕐ℤ

Logic

⊥⊤∧⋀∨⋁∀∃∄¬○□◇

Operator

↾⋄∩⋂∪⋃⊔⨆⊓⨅∖∝⊎⨄±∓×÷⋅⋆∙∘⊕⨁⊗⨂⊙⨀⊖⊘∑∏∐∫∮¯⨿℧⤜⪢⨡⧹⁀

Punctuation

∷⟨⟩⟪⟫⌈⌉⌊⌋⦇⦈⟦⟧⦃⦄⦉⦊«»¦⫿…⋯‐·⨾⦁′‹›

Relation

⊢⊨⊩⊫⊣√≤≥≪≫≲≳⪅⪆∈∉

⊂⊃⊆⊇⊏⊐⊑⊒≠∼≐≃≈≍≅⌣≡

⌢≺≻≼≽∥‖⫴⫽⊲⊳⊴⊵◃▹△≜

≀◁⩤▷⩥⦂⋿

Unsorted

⋈⨝†‡∞♣♢♡♠ℵ∅∇∂♭♮∠©

®ªº§¶¡¿£¥¢¤°◊℘´ı

¨¸˝ϵ⌑⏎✓✗⌂❖

End
\end{lstlisting}

\noindent 
The language definition has some limitations, discussed below. 

\section{Credit}
This version of isabelle-listings was produced by John Hughes, but about 90 percent of it comes
from the work of Jens Christoph Buerger, whose efforts I appreciate a great deal. 
Much credit also goes to various helpful folks on tex.stackexchange.com who helped 
me understand just enough about the challenges of dealing with character codes to 
get to the point where all this works. 

\section{Using this language definition}
As in the example above, you merely open a listing, paste in stuff from the
Isabelle/jEdit window, and close the listing.

\subsection{Not the way to do it!}
For most uses of Isabelle, this is the \textbf{wrong} way to produce source-code
listings. The \textbf{right} way is to insert some magic instructions into the 
source code, use Isabelle's own build system to produce a LaTeX file, and then 
include `snippets' from that file into whatever document you're producing. 
This is documented in the Isabelle build system manual, although as of a few 
months ago the `comment' style shown there is no longer in use, and there are a 
couple of typos -- excess backslashes -- that make things fail. 

That system seems complex --- it is! --- but it has the advantage that as your 
source code changes, so does your document. It has the disadvantage that by 
default it won't want to produce output for theories with errors in them, or may 
(depending on option-settings) suppress things with `sorry' in them. If you're 
trying (as I am) to guide beginning users in the use of Isabelle, and want to 
demonstrate something being done wrong (and its correction!), this is actually 
somewhat annoying. And if you're writing a paper comparing multiple proof 
systems, having the Isabelle build-system in the middle of your document 
may be an annoyance rather than an aid. 

The other disadvantage is that the LaTeX generated by the Isabelle build 
system isn't particularly readable: many characters are replaced by a LaTeX 
macro like \verb|\isasymW| for a "mathcal" letter W, etc. The meanings of 
these macros is defined in a style file, and that results in a 
beautifully-formatted result, but hard-to-read LaTeX input. Of course, the 
notion is that you shouldn't be messing with that input at all once Isabelle 
produces it, but that's not how the document-preparation cycle always 
progresses for me. 

\section{Limitations}
This project has several limitations.
\begin{itemize}
    
\item  This project was developed using Isabelle2024 in early 2025. Isabelle 
 will no doubt gradually add more symbols, and this will become obsolete. 

\item There are some symbols that cannot be used because they will appear as 
odd blocks in the output rather than as the symbol you wanted. The font I've 
chosen to use (Julia) seems very well provisioned. If you try to shift to 
a different font, you may find that some symbols are not available. 

\item For reasons I don't understand, the Euro symbol (€) causes problems.

\item Using this language-definition requires using LuaLaTeX or XeTeX (although 
I haven't tested it there) rather than pdfLaTeX.

\item Using this language definition requires changing the input-characters 
allowed in your document. Many if not most of the unusual characters used in the jedit 
window have character codes outside of the normal range traditionally allowed by LaTeX. Some
are in the range range 1000-2000, which may not be supported by all editors.

\item As a particular example, Overleaf's current editor cannot handle many 
of these characters, so this language definition won't work there.

\item Many characters in Isabelle's own style-definition require shifting 
vertically to look good. Almost all the arrows, for instance, are shifted 
somewhat. This language-definition does none of that.

\item Isabelle uses arrows with many different lengths; this font does a 
bad job distinguishing among those. 

\item This language-definition is unsupported. I need it for a particular
project, but do not anticipate maintaining it in the future.

\item The failure of the `listing` package to properly handle unicode without
the whole `\verb|lst@DefEC|' thing appears to be a bug. If the author of
`listings' fixes that bug, this language definition can be considerably 
simplified. I may actually do so if that fix is made. 

\end{itemize}



\end{document}